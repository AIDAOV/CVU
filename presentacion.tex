\documentclass{beamer}
\mode<presentation>
{
  %\usetheme{Amsterdam}
  %\usetheme{Berlin}
  %\usetheme{Copenhagen}
  %\usetheme{Malmoe}
  %\usetheme{Dresden}
  \usetheme{Darmstadt}
  %\usetheme{Frankfurt}
  %\usetheme{Copenhagen}
  %\usetheme{Singapore}
  %\usetheme{Hannover}
  %\usetheme{Berkeley}
  %\usetheme{Montpellier}
  %\usetheme{Pittsburgh}
  %\usetheme{Warsaw}
  % or ...
  \setbeamercovered{transparent}
  % or whatever (possibly just delete it)
}
\usepackage[utf8]{inputenc}
\usepackage{times}
\usepackage[T1]{fontenc}
\usepackage{pstricks}
\usepackage{fancyvrb}
\title[cvu] % (optional, use only with long paper titles)
{Conversión automática de CV a CVU}
\subtitle{} % (optional)
\author{W. Luis Mochán Backal}
\institute[ICF-UNAM] % (optional, but mostly needed)
{
  Instituto de Ciencias Físicas, UNAM\\
}
\date[2017-07-14, CONACyT]
{Consejo Nacional de Ciencia y Tecnología\\CONACyT\\CdMx, agosto 14, 2017}
\subject{CVU}
% This is only inserted into the PDF information catalog. Can be left
% out. 
\pgfdeclareimage[height=.8cm]{logounam}{logos}
\logo{\pgfuseimage{logounam}}
\DeclareGraphicsRule{*}{mps}{*}{}
\newif{\ifskip}
\begin{document}
\begin{frame}
  \titlepage
\end{frame}
\begin{frame}{Mi currículum en pdf}
  \includegraphics<1>[width=\textwidth]{curriculum1}  
  \includegraphics<2>[width=\textwidth]{curriculum2}  
\end{frame}
\begin{frame}[fragile]{Mi currículum en latex\ldots}
\begin{Verbatim}[fontsize=\small]
\documentclass{article}
\usepackage[utf8]{espanhol}
\addtolength{\textwidth}{2in}
\addtolength{\hoffset}{-1in}
\usepackage{fancyhdr}
\lhead{\small Curriculum Vitae, W. Luis Mochán, \fecha}
\rhead{\small \leftmark}
\pagestyle{fancy}
\usepackage{multicol}
\usepackage{comment}
%\excludecomment{examendegrado}
%\newcommand{\est}[1]{{\bf #1}} %indica estudiantes
\newcommand{\est}[1]{#1} %no hhagas nada para indicar estudiantes
\renewcommand{\sectionmark}[1]{\markboth{\thesection. #1}{}}
\end{Verbatim}
\end{frame}
\begin{frame}[fragile]{Mi currículum en latex\ldots}
\begin{Verbatim}[fontsize=\small]
\subsection{Artículos de investigación en revistas periódicas}
\begin{milista}
\addtocounter{footnote}{1}\footnotetext{Con arbitraje}
\setcounter{staticfoot}{\value{footnote}}
\foo{\it  Optical Properties of Quasi-Two Dimensional Systems:
	Non Local Effects\/},
	W. Luis Mochán y Rubén Barrera,
	Physical Review B \vol23, 5707-5718, (1981).
\foo{\it  Surface Contribution to the Optical Properties of
	Non-Local Systems\/},
	W. Luis Mochán, Ronald Fuchs y Rubén G. Barrera,
	Physical Review B \vol27, 771-780  (1983).
\foo{\it  Surface Local-Field Effect\/},
	W. L. Mochán y R.G. Barrera,
	Journal de Physique (París) \vol45 C5, 207-212 (1984).
\end{Verbatim}
\end{frame}
\begin{frame}[fragile]{Precondición}
\begin{Verbatim}[fontsize=\small]
.Artículos
  [\foo{\it  [Optical Properties of Quasi-Two Dimensional
        Systems:Non Local Effects]\/}, [[W. Luis Mochán]
      y [Rubén Barrera],] [Physical Review B] \vol[23],
      [5707-5718], ([1981]).] 
  [\foo{\it  [Surface Contribution to the Optical Properties
        of Non-Local Systems]\/}, [[W. Luis Mochán], [Ronald
        Fuchs] y [Rubén G. Barrera]], [Physical Review B]
        \vol[27], [771-780]  ([1983]).]
\end{Verbatim}
\vdots
\begin{Verbatim}[fontsize=\small]
.FinArtículos
\end{Verbatim}
\end{frame}
\begin{frame}[fragile]{Formato}
  \begin{itemize}
  \item Formato casi libre pero estructurado:\\
    .Artículos\\
    \ldots{}[\ldots{}[Título]\ldots\\
      {}[\ldots{}[Autor 1]\ldots {}[Autor 2]\ldots]\ldots\\
      {}[Revista]\ldots\\
      {}[Volumen]\ldots{}[Página(s)]\ldots{}[Año]\ldots]\\
    \ldots{}[{\em otro artículo}] [{\em y otro}]\ldots\\
    .FinArticulos
  \item '\ldots' representa basura (comandos de latex)
  \item Espacios múltiples, tabuladores, cambios de línea se ignoran.
  \item Cada artículo, cada lista, cada campo, entre corchetes. Lo que
    esté fuera de corchetes se ignora.
  \end{itemize}
\end{frame}
\begin{frame}{Tiempos}
  Tardé 22 minutos en precondicionar 34 artículos usando emacs, pero
  sin usar macros.
  
\end{frame}
\begin{frame}[fragile,shrink=10]{Extracción automática a YAML}
  Programa articulos2yaml.pl
  \begin{Verbatim}[fontsize=\small]
#!/usr/bin/env perl
use warnings;
use strict;
use feature qw(say);
use utf8;
use Text::Balanced qw(extract_bracketed);
use YAML::Tiny;
use open qw/ :std :encoding(utf8) /;

my $yaml=YAML::Tiny->new();
my $articulos;
while(<>){
    $articulos.=$_ if(/^\.Artículos/../^\.FinArtículos/);
}
my @articulos;
while(my $articulo=extract_bracketed($articulos,'[]','(?ms:.*?)(?=\[)')){
    push @articulos, $articulo;
}
  \end{Verbatim}
\end{frame}
\begin{frame}[fragile,shrink=10]{Uso}
  \begin{Verbatim}[fontsize=\small]

    $ # Para extraer artículos y mostrarlos en
    $ # la pantalla:
    $ ./articulos2yaml.pl precond.txt
    $ #
    $ #
    $ # Para extraer artículos y guardarlos en
    $ # un archivo
    $ ./articulos2yaml.pl precond.txt > arts.yaml
  \end{Verbatim}
\end{frame}
\begin{frame}[fragile,shrink=10]{Salida}
\begin{Verbatim}[fontsize=\small]
artículos:
  -
    autores:
      - 'W. Luis Mochán'
      - 'Rubén Barrera'
    año: '1981'
    paginas: 5707-5718
    revista: 'Physical Review B'
    titulo: 'Optical Properties of Quasi-Two Dimensional Systems: Non Local Effects'
    volumen: '23'
  -
    autores:
      - 'W. Luis Mochán'
      - 'Ronald Fuchs'
      - 'Rubén G. Barrera'
    año: '1983'
    paginas: 771-780
    revista: 'Physical Review B'
    titulo: 'Surface Contribution to the Optical Properties of Non-Local Systems'
    volumen: '27'
  -
\end{Verbatim}
\end{frame}
\begin{frame}{YAML}
  \begin{itemize}
  \item   La salida es un archivo de texto en el formato YAML.
  \item YAML es un formato de texto muy estructurado.
  \item Permite una fácil inspección y edición por humanos.
  \item Permite un fácil proceso por computadoras.
  \end{itemize}
\end{frame}
\begin{frame}[fragile]{Errores}
  \begin{itemize}
  \item Ejemplo de una salida errónea.
  \end{itemize}
    \begin{Verbatim}[fontsize=\small]
  -
    autores:
      - 'Raul Alfonso Vázquez Nava'
      - 'Marcelo del Castillo-Mussot'
      - 'W. Luis Mochán'
      - 'Physical Review B'
      - '47'
      - '3971'
      - '1993'
    año: '5'
    paginas: 'Journal of Physics: Condensed Matter'
    revista: '\foo{\it[ Effective dielectric response of a composite with aligned spheroidal inclusions],} [Rubén G. Barrera],[ J. Giraldo ]y [W. Luis Mochán], [Physical Review B] \vol[ 47],[ 8528] ([1993]). '
    titulo: 'Reflectance of non-local conducting superlattices'
    volumen: '\foo{\it [Infrared anisotropy as a surface probe],} [ W. Luis Mochán] y [José Récamier]'
  \end{Verbatim}
\end{frame}
\begin{frame}[fragile]{Errores}
  \begin{itemize}
  \item Causa del error: Corchetes mal balanceados.
  \begin{Verbatim}
\foo{\it [[Reflectance of non-local conducting superlattices],}
\est{[[Raul Alfonso Vázquez Nava]}, [Marcelo del Castillo-Mussot] y[ W.
Luis Mochán],[
Physical Review B] \vol[ 47], [3971] ([1993]).
]
[\foo{\it[ Effective dielectric response of a composite with aligned
spheroidal inclusions],} [Rubén G.  Barrera],[ J. Giraldo ]y [W. Luis
Mochán], [Physical Review B] \vol[ 47],[ 8528] ([1993]).  
]
  \end{Verbatim}
\item Detección y corrección de todos los errores en condicionar 34
  artículos : Menos de 5 mins.
  \end{itemize}
\end{frame}
\begin{frame}[fragile]{Base de datos}
  Programa yaml2db.pl
  \begin{Verbatim}
$ # Crea y llena una base de datos
$ ./yaml2db.pl -y arts.yaml -d arts.db
$
$ # Tablas
$ sqlite3 arts.db '.schema'
CREATE TABLE artículos(título,autores,revista,año,
    páginas,volumen);
CREATE TABLE autores (autor,autorcanónico);
CREATE TABLE revistas(revista,revistacanónica);
  \end{Verbatim}
\end{frame}
\begin{frame}[fragile,shrink=25]{Lista de artículos, revistas, autores}
  \begin{Verbatim}[fontsize=\small]
$ sqlite3 arts.db '.headers on' 'select ROWID,* from artículos'
rowid|título                    |autores|revista |año|páginas  |volumen
1    |Optical Properties of...  |1,2    |1      |1981|5707-5718|23
2    |Surface Contribution to...|1,3,4  |1      |1983|771-780  |27
3    |Surface Local-Field Effect|5,6    |2      |1984|207-212  |45 C5
.
$ sqlite3 arts.db '.headers on' 'select ROWID,* from revistas'
rowid|revista                    |revistacanónica
1    |Physical Review B          |Physical Review B
2    |Journal de Physique (París)|Journal de Physique (París)
3    |Physical Review Letters    |Physical Review Letters
.
$ sqlite3 arts.db '.headers on' 'select ROWID,* from autores'
rowid|autor         |autorcanónico
1    |W. Luis Mochán|W. Luis Mochán
2    |Rubén Barrera |Rubén Barrera
3    |Ronald Fuchs  |Ronald Fuchs
.
  \end{Verbatim}
\end{frame}
\begin{frame}[fragile]{Canoniza revistas, autores}
  \begin{itemize}
  \item Tabla de revistas: mapea nombres comunes en números y en nombres
    canónicos. 
  \item Tabla de autores: mapea nombres comunes en números y en nombres
    canónicos. 
  \end{itemize}
  \begin{Verbatim}
$ # Asocia nombre común con nombre oficial
$ sqlite3 arts.db 'UPDATE revistas 
   SET revistacanónica="Physical Review Letters"
   WHERE revista LIKE "PRL"'
$ #
$ sqlite3 arts.db 'UPDATE autores 
   SET autorcanónico="Rubén G. Barrera-Pérez"
   WHERE autor LIKE "Rubén Barrera"'
  \end{Verbatim}
\end{frame}
\begin{frame}[fragile]{Añade campos}
  \begin{itemize}
  \item Se puede añadir información adicional a cada tabla,
    información que no está en la lista de publicaciones.
  \end{itemize}
  \begin{Verbatim}
$ # Añade issn y país a la tabla de revistas
$ # (no a la de artículos)
$ sqlite3 arts.db 'ALTER TABLE revistas ADD issn'
$ sqlite3 arts.db 'ALTER TABLE revistas ADD país'
$
$ # Asigna issn a alguna revista
$ sqlite3 arts.db 'UPDATE revistas
   SET issn="0031-9007" WHERE
   revistacanónica LIKE "Physical Review Letters"'
$ sqlite3 arts.db 'UPDATE revistas
   SET país="Estados Unidos" WHERE
   revistacanónica LIKE "Physical Review Letters"'
  \end{Verbatim}
\end{frame}
\begin{frame}[fragile,shrink=10]{Formato APA}
  \begin{itemize}
  \item El programa \Verb+apa2yaml.pl+ $\sim$~50 líneas) procesa artículos
    en el formato de la American Psychological Association. 
  \item P. ej., convierte:
  \begin{Verbatim}[fontsize=\small]
Agranovich, V. M., & Gartstein, Y. N. (2009). Electrodynamics of
metamaterials and the Landau–Lifshitz approach to the magnetic
permeability. Metamaterials, 3(1),
1–9. https://doi.org/10.1016/j.metmat.2009.02.002
  \end{Verbatim}
\item en
  \begin{Verbatim}[fontsize=\small]
  -
    autores:
      - 'V. M. Agranovich'
      - 'Y. N. Gartstein'
    año: '2009'
    páginas: 1–9
    revista: 'Metamaterials, '
    título: 'Electrodynamics of metamaterials and
             the Landau–Lifshitz approach to the magnetic permeability'
    volumen: '3'
  \end{Verbatim}
  \end{itemize}
\end{frame}
\begin{frame}[fragile]{Formato NATURE}
\begin{itemize}
\item El programa \Verb+nature2yaml.pl+ ($\sim$50 líneas) procesa artículos
  en el formato recomendado por Nature. 
  \item P. ej., convierte:
  \begin{Verbatim}[fontsize=\small]
3.Guenneau, S., Zolla, F. & Nicolet, A. Homogenization of 3D finite
photonic crystals with heterogeneous permittivity and
permeability. Waves in Random and Complex Media 17, 653–697 (2007).
  \end{Verbatim}
\item en
  \begin{Verbatim}[fontsize=\small]
   -
    autores:
      - 'S. Guenneau'
      - 'F.  Zolla'
      - 'A.  Nicolet'
    año: '2007'
    páginas: 653–697
    revista: 'Waves in Random and Complex Media'
    título: 'Homogenization of 3D finite photonic crystals with heterogeneous permittivity and permeability'
    volumen: '17'

  \end{Verbatim}
  \end{itemize}
\end{frame}
\begin{frame}[fragile]{Conclusiones}
  \begin{itemize}
  \item Con muy poco esfuerzo se puede añadir estructura a un CV
    ordinario para automatizar su proceso.
  \item Hice un programa \Verb|articulos2yaml.pl| que convierte la
    lista de artículos de un CV del formato {\em título, autores,
      revista volumen, páginas (año)} a YAML.
  \item Hice programas análogos \Verb+nature2yaml.pl+ y
    \Verb+apa2yaml.pl+ que traducen automáticamente del 
    formato recomendado por APA y por Nature a YAML.
  \item Otros formatos son triviales de añadir.
  \item El formato YAML es totalmente estructurado pero legible por
    humanos, por lo que permite detectar y corregir errores 
    manualmente, 
  \item y se puede vertir a una base de datos automáticamente.
  \item Mis programas son sólo {\em demostraciones de concepto}. Se
    pueden mejorar y extender.
  \end{itemize}
\end{frame}
\begin{frame}[fragile]{Conclusiones}
  \begin{itemize}
  \item Hice un programa \Verb|yaml2db.pl| para hacer una base de
    datos \verb|sqlite3|.
  \item Con comandos SQL se puede manipular la base de datos agregándo
    nombres canónicos y campos adicionales, evitando repeticiones.
  \item Estas bases de datos podrían importarse automáticamente a los
    CVU's de CONACyT.
  \item Las pocas labores repetitivas requeridas (p. ej., añadir
    corchetes al CV y canonizar nombres) pueden llevarse a cabo
    rápidamente por un capturista o personal secretarial con un mínimo
    de capacitación.
  \end{itemize}
\end{frame}
\begin{frame}[fragile]{Ligas relacionadas}
  \begin{itemize}
  \item Sistema de reportes \Verb+https://wlmb.github.io/CVU+
  \item Lista de correos \Verb+http://em.fis.unam.mx/pipermail/cvu+
  \item Convertidor CV a YAML
    \Verb+http://bit.ly/2tUieuF+
  \item Convertidor APA a YAML
    \Verb+http://bit.ly/2tWXOND+
  \item Convertidor Nature a YAML
    \Verb+http://bit.ly/2f5SaWF+
  \item Convertidor YAML a base de datos
    \Verb+http://bit.ly/2eYkdHT+
  \item Esta presentación
    \Verb+http://bit.ly/2tOunNH+
  \item W. Luis Mochán \Verb+mailto:mochan@fis.unam.mx+
  \end{itemize}
\end{frame}
\end{document}
